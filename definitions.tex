% Need to use CMYK to get preview.app to show colors correctly
\definecolor{accentColor}{cmyk}{0.222, 0.111, 0, 0.435}  % RGB: 112, 128, 144
\def\dateColWidth{0.125\textwidth}
\newlength{\sectionHeaderIndent}
\setlength{\sectionHeaderIndent}{2.25cm}
\newlength{\sectionHeaderIndentOffset}
\setlength{\sectionHeaderIndentOffset}{0.1cm}
\newlength{\datedEntryPadding}
\setlength{\datedEntryPadding}{0.1cm}
\def\spaceAfterSection{\vspace{0.05 in}\break}
\def\tabularIndent{0.095\textwidth}
\def\skillsColWidth{0.33\textwidth}
\def\divLine{
    \vspace{0.2cm}
    \hrule
    \vspace{0.2cm}
}
\def\divNoLine{

    \vspace{0cm}

}
\newcolumntype{Y}{>{\centering\arraybackslash}X} % For centered tabluarx column

\newcommand{\mySubsection}[1] {
    {\large\textbf{#1}}
}

\newcommand{\subtitledSubsection}[2] {
    {\large\textbf{#1}} -- {\textit{#2}}
}

\newcommand{\sectionHeader}[1]{
    \vspace{0.1cm}
    \colorbox{accentColor}{%
        \rlap{%
            \hspace{\sectionHeaderIndent}%
            \hspace{\sectionHeaderIndentOffset}%
            \hspace{-\fboxsep}%
            \Large{\lsstyle\textbf{\color{White}#1}}%
        }%
        \hspace{\linewidth}\hspace{-2\fboxsep}%
    }
}

\newcommand{\datedEntry}[2]{
    \begin{minipage}[c]{\sectionHeaderIndent}
        % Date
        \raggedright\color{accentColor}{#1}
    \end{minipage}%
    \begin{minipage}[c]{\datedEntryPadding}
        % Spacing between date and body
        \hspace{0cm}
    \end{minipage}%
    \begin{minipage}[c]{\textwidth - \sectionHeaderIndent - \datedEntryPadding}
        % Body
        #2
    \end{minipage}
}

\newcommand{\datedEntryWithLink}[3]{
    \href{#3}{
    \begin{minipage}[c]{\sectionHeaderIndent}
        % Date
        \raggedright\color{accentColor}{#1}
    \end{minipage}%
    \begin{minipage}[c]{\datedEntryPadding}
        \hspace{0cm}
    \end{minipage}%
    \begin{minipage}[c]{\textwidth - \sectionHeaderIndent - \datedEntryPadding}
        % Body
        #2
    \end{minipage}
    }
}

\newcommand{\skillHeader}[1]{
    \textbf{#1} --
}

\newcommand{\tag}[1] {
    % Prints a 'tag'
    %\colorbox{Black!75}{\fontsize{8 pt}{0}{\textsf{\color{White}\uppercase{#1}}}}%
    \mbox{\fontsize{8 pt}{0}{\textsf{\color{Black}\uppercase{#1}}}}%
}

\newcommand{\tagl}[1] {
    % Prints a 'tag'
    %\colorbox{Black!75}{\fontsize{8 pt}{0}{\textsf{\color{White}\uppercase{#1}}}}%
    \mbox{\fontsize{8 pt}{0}{\textsf{\color{Black}\uppercase{#1}}}}%
    \ \vline\ %
}

\newcounter{i}

\newcolumntype{C}[1]{>{\centering\let\newline\\\arraybackslash\hspace{0pt}}m{#1}} % Table column type that lets you set a fixed column width
\newcolumntype{L}[1]{>{\raggedright\let\newline\\\arraybackslash\hspace{0pt}}m{#1}} % Table column type that lets you set a fixed column width
